\documentclass{bioinfo}
\copyrightyear{2013}
\pubyear{2013}

\usepackage{natbib}
\bibliographystyle{apalike}

\DeclareMathOperator*{\argmax}{argmax}

\begin{document}
\firstpage{1}

\title{Aligning sequence reads, clone sequences and assembly contigs with BWA-MEM}

\author[Li]{Heng Li$^{1,}$\footnote{to whom correspondence should be addressed}}

\address{$^1$Broad Institute, 7 Cambridge Center, Cambridge, MA 02142, USA}

\history{Received on XXXXX; revised on XXXXX; accepted on XXXXX}
\editor{Associate Editor: XXXXXXX}
\maketitle

\begin{abstract}
\section{Summary:} BWA-MEM is a new alignment algorithm for aligning sequence
reads or long query sequences against a large reference genome such as human.
It automatically chooses between local and end-to-end alignments, supports
paired-end reads and performs split alignment. The algorithm is robust to
sequencing errors and applicable to a wide range of sequence lengths from 70bp
to a couple of megabases. For short-read mapping, BWA-MEM shows better
performance than many state-of-art read aligners to date.

\section{Availability and implementation:} BWA-MEM is implemented as a
component of BWA, which is available at http://github.com/lh3/bwa.

\section{Contact:} hengli@broadinstitute.org
\end{abstract}

\section{Introduction}

Most short-reads mappers for next-generation sequencing (NGS) data were developed
when sequence reads were about 36bp in length. For short reads, it is reasonable
to require end-to-end alignment (i.e. every read base to be aligned to the
reference) and to only report hits within certain hamming or edit distance.
However, with emerging technologies and improved chemistry, NGS reads are not
short any more, which poses new challenges to read alignment. For longer reads,
it becomes more important to allow long gaps under the affine-gap penalty and
to report multiple non-overlapping local hits potentially caused by structural
variations or misassemblies in the reference genome. Many short-read alignment
algorithms are not applicable or not preferred for mapping longer reads. At the
same time, although several mature algorithms exist for aligning capillary
sequence reads, they are slow and lack features for analyzing large-scale NGS
data. Fast moving NGS technologies keep pressing for the development of new
alignment algorithms.

In this background, a few long-read alignment algorithms, including BWA-SW~\citep{Li:2010fk},
Bowtie2~\citep{Langmead:2012fk}, Cushaw2~\citep{Liu:2012uq} and GEM~\citep{Marco-Sola:2012kx}, have been developed. However, they all have some
weakness. BWA-SW is slower than bowtie2 at a comparable accuracy and less
accurate then Cushaw2 at a comparable speed (see Results). Bowtie2 and Cushaw2
do not work with very long query sequences. These three aligners do not fully
use the paired-end information to improve alignments in hard regions, either.
While GEM is both fast and accurate, it mandates end-to-end alignment and does
not perform affine-gap alignment, which may limit its uses for long-read
alignment. A fast, accurate and fully featured aligner accepting sequences
with a wide range of lengths is still lacking. This motivates us to explore
a new algorithm.

\begin{methods}

\section{Methods}

\subsection{Aligning single query sequence}

\subsubsection{Seeding and re-seeding} BWA-MEM follows the canonical
seed-and-extend paradigm. It initially seeds an alignment with supermaximal
exact matches (SMEMs) using an algorithm we found previously~\citep[Algorithm 5]{Li:2012fk}. This algorithm
essentially finds at each query position the longest exact match covering the
position. Suppose we have a SMEM of length $l$ with $k$ occurrences in the
reference genome. If $l$ is too large (over 28bp by default), we then re-seed
with exact matches that cover the middle base of the SMEM and occur at least
$k+1$ times in the genome, which can be done by requiring a minimum occurrence
in the original SMEM algorithm. Re-seeding reduces the chance of missing a seed
from the true alignment.

\subsubsection{Chaining and chain filtering} We call a group of seeds that are
colinear and close to each other as a \emph{chain}. While seeding, we chain
the seeds with a naive algorithm. We then filter out short chains that are
largely contained in a long chain and is much worse than the long chain (by
default, both 50\% and 38bp shorter than the long chain). Note that chain
filtering aims to reduce unsuccessful seed extension at a later step. The
chains detected here do not need to be accurate. We may implicitly split or
merge chains during seed extension.

\subsubsection{Seed extension} We rank the seeds by the length of the chains
the seeds belong to and then by the length of the seeds in each chain. We start
from the highest ranked seed and for each seed, we drop the seed if it is
already contained in an alignment found before, or extend with a
affine-gap-penalty dynamic programming (DP) otherwise. We will find all the
potential alignments of a query sequence after this step.

BWA-MEM's seed extension differs from the standard seed extension in two
aspects. Firstly, suppose at a certain extension step we come to reference
position $x$ with the best extension score achieved at query position $y$.
BWA-MEM will stop extension if the score at $(x,y)$ is smaller than the best
extension score observed so far by $Z+|x-y|\times p_{\rm gapExt}$, where
$p_{\rm gapExt}$ is the gap extension penalty and $Z$ is an arbitrary value. This heuristic avoids extending
through a poorly aligned region with good flanking alignment. It is similar
to the X-dropoff heuristic in BLAST~\citep{Altschul:1997vn}, but does not penalize long gaps in
one of the reference or the query sequences.

Secondly, while extending a seed, BWA-MEM tries to keep track of the best
extension score reaching the end of the query sequence. If the difference
between the best score reaching the end and the best local alignment score is
below a threshold, which is called clipping penalty, the local alignment will
be rejected even if it has a higher score. BWA-MEM uses this strategy to
automatically choose between local and end-to-end alignments. It may allow true
substitutions towards the end of a read and thus reduces reference bias, while
avoids introducing excessive mismatches and gaps which may happen when we force
an end-to-end alignment for a read bridging a break point associated with a
long insertion or deletion (indel) or a structural variation or a misassembly.

\begin{figure}[tb]
\centering
\includegraphics[width=0.4\textwidth]{eval/alnroc-se}\\
\includegraphics[width=0.4\textwidth]{eval/alnroc-pe}
\caption{Percent mapped reads as a function of the false alignment rate under
different mapping quality cutoff. $10^6$ pairs of 101bp reads are simulated
from the human reference genome using wgsim (http://bit.ly/wgsim2) with 1.5\%
substitution errors and 0.2\% indel variants. The insert size follows a normal
distribution $N(500,50^2)$. The reads are aligned back to the genome either as
single end (SE; top panel) or as paired end (PE; bottom panel). GEM is configured to
allow up to 5 gaps and to output suboptimal alignments (option `--e5 --m5 --s1' for SE and `--e5 --m5 --s1 --pb' for PE). Its
mapping quality is estimated with a BWA-like algorithm. Other mappers are run
essentially with the default setting except specifying the insert size
distribution. The run time on a single CPU core is shown in the parantheses.}\label{fig:eval}
\end{figure}.

\subsection{Paired-end mapping}

\subsubsection{Estimating insert size distribution and rescuing missing hits}
Like BWA, BWA-MEM processes a batch of reads at a time.  Given paired-end data,
it estimates the mean $\mu$ and the variance $\sigma^2$ of the insert size
distribution from reliable single-end alignments. For the top 100 hits (by
default) of each end, BWA-MEM performs the Smith-Waterman alignment (SW) for
its mate within a window $[\mu-4\sigma,\mu+4\sigma]$ from the hit. The second
best alignment score is recorded to detect potential mismapping in a long
tandem repeat.  Hits found both from the single-sequence alignment and SW
rescuing will be fed to the next step for pairing.

\subsubsection{Pairing} Given the $i$-th hit for the first read and $j$-th hit
for the second, BWA-MEM computes their distance $d_{ij}$ if the two hits are in
the right orientation, or sets $d_{ij}$ to infinity otherwise. It scores the
pair $(i,j)$ as $S_{ij}=S_i+S_j-\min\{-a\log_4 P(d_{ij}),U\}$, where $S_i$ and
$S_j$ are the SW score of the two reads in the pair, respectively, $a$ is the
matching score and $P(d)$ gives the probability of observing an insert size
larger than $d$ assuming a normal distribution. `$\log_4$' arises when we
relate SW score to a probabilistic model. $U$ is the chimeric penalty,
which controls preference of pairing: if $d_{ij}$ is small enough such
that $-a\log_4 P(d_{ij})<U$, BWA-MEM prefers to pair the two ends; otherwise
it prefers the unpaired alignments. BWA-MEM chooses the pair $(i,j)$ that
maximizes $S_{ij}$ as the final alignment for both ends.
This pairing strategy jointly considers the alignment scores of both ends,
the insert size and the possibility of chimeric read pairs.

\end{methods}

\section{Results and Discussions}

We evaluated the performance of BWA-MEM on simulated data together with
GEM, Bowtie2, Cushaw2, BWA-SW, BWA~\citep{Li:2009uq} and NovoAlign
(http://novocraft.com) (Figure~\ref{fig:eval}). On accuracy, NovoAlign
is the most accurate. BWA-MEM is close to NovoAlign for PE reads
and is comparable to GEM and Cushaw2 for SE. On speed, BWA-MEM is similar to
GEM and Bowtie2 for this data set, but is about 6 times as fast as Bowtie2
for a 650bp long-read data set.

To evaluate the performance for even long query sequences, we aligned the {\it E. coli}
K-12 MG1655 substrain (AC:NC\_000913) against the 536 strain (AC:NC\_008253) with 
both BWA-MEM and nucmer~\citep{Kurtz:2004zr}. Nucmer finished the alignment in 25 seconds, giving
105,505 substitution differences between the strains. BWA-MEM is slower.
It finished the alignment in 131 seconds, reporting 104,321 substitutions of which 102,241
overlapping with the nucmer output. Manual examination of the
substitutions unique to each aligner reveals that most of them fall in short regions
with very high divergence. It is unclear which aligner is better. Although
Nucmer is faster in the evaluation, only BWA-MEM scales to mammalian genomes.

BWA-MEM is a fast and accurate alignment algorithm for sequence reads and is
one of the few aligners that work efficiently with both short reads and long sequences up
to a few megabases. It is still possible to greatly accelerate BWA-MEM for long
sequences by using SSE2-powered banded DP and by restricting DP to regions not
covered by a long seed.

\section{Acknowledgements}
\paragraph{Funding\textcolon} NIH 1U01HG005208-01.
\bibliography{bwamem}

\end{document}
